\begin{foreignabstract}
This thesis presents an overview about Player Modeling and shows how Provenance can help while building such models.

Player Modeling is inherently associated with games in general. In this context, some questions come to the mind when modeling a player - the foremost question is: what is a player model? A second possible question can be: what kind of input is used to model a player?... going on: what processes are involved while building a player model? A possible question to close the loop: what is the purpose\textbackslash{}outcome of building player models after all?

This work was conceived because the author has a true appreciation for electronic games and specifically in how mapping\textbackslash{}modeling player motivations, preferences and actions can change the outcome for both players and game creators\textbackslash{}developers.

We live in the big data boom era and as such we have plenty of raw data to explore. Game session logs can be explored in many different ways. This data hides gems that can improve player entertainment while maximizing game creators budget.

All in all what the player wants is a engaging game with balanced characteristics, that is, not too easy and not too hard; not too short and not too long, etc. At the same time, game developers want to create those engaging games of course. To be able to create such games, the use of player models can be of high value and that's what this thesis discuss in a more deep level.
\end{foreignabstract}