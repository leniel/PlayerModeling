\chapter{Introduction}

Player Modeling is currently a very relevant topic in game Artificial Intelligence (AI) research.\abbrev{AI}{Artificial Intelligence}

The term “player model” is nebulous, and a broad spectrum of possible definitions have been used in various publications \citep{SmithInclusiveTaxonomyPlayer}. While many people have described player modeling and discussed its importance, the definition and usage of the idea has grown over time, with no single clear definition or scope.

The practice of trying to understand players through analysing their interactions with the games they play is becoming more and more common and important. Under the labels “player modeling”, “game data mining” and “game analytics", researchers from various fields (game studies, game design, computational intelligence and statistical machine learning) are bringing learning algorithms and statistical techniques to bear on the logs of (usually multiple, sometimes huge numbers of) players playing games.

The most important goal for almost every existing game is entertainment \citep{NareyekAIComputerGames2004} and this is achieved with a combination of items like graphics, story (in some game genres) and AI.
Entertainment is a subjective concept and, in order to know how much a game entertains a player, some general metrics are evaluated. One of the most important metrics is immersion, which is generally related to how absorbing and engaging a game is. Two common approaches to achieve immersion are the use of stunning graphics and the development of a good AI system. While graphics are responsible for initially “seducing” the players, AI is responsible for keeping them interested and engaged in the game.
The aesthetic elements were the industry’s main concern for a long time since they are more attractive. It's easier to convince a customer to buy a game because it “looks good” than because it has a complex AI that is much harder to be presented in marketing events. However this scenario is starting to change. Game AI is receiving increasing attention since it helps keeping the player engaged to the game.

The successful design of player experience can make the difference between a game that engages players for extended periods of time and a game that contains content — whether that is mechanics, levels, maps or narrative plots — that fail to elicit appropriate experience patterns and rich affective experiences to its users \citep{YannakakisEmotionGames2011}.

The primary goal of player modeling and player experience research is to understand how the interaction with a game is experienced by individual players. Thus, while games can be utilized as an arena for eliciting, evaluating, expressing and even synthesizing experience, one of the main aims of the study of players in games is the understanding of players’ cognitive, affective and behavioral patterns. Indeed, by the nature of what constitutes a game, one cannot dissociate games from player experience \citep{YannakakisPlayerModeling2013}. \cite{MunozPlayPhysicsEmotionalGames2010}

In summary, player modeling is the study of computational means for the modeling of player cognitive, behavioral, and affective states which are based on data (or theories) derived from the interaction of a human player with a game \citep{YannakakisGameAIRevisited2012}.
