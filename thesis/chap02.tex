\chapter{Games}
In this work player modeling is applied to electronic games. What follows is a brief background information about what constitutes a game.

\section{What is a Game?}
The existence of a discussion about what a game is, or whether some activity is actually a game may surprise those who stumble upon it for the first time. However, while studying games, it is important to understand what we are analyzing or what we are creating. For this, it is better that the concept of what defines a \enquote{game} is plain clear and that we know how far we are moving away from an \enquote{ideal game} when we are creating a new game \citep{XexeoGeraldoquesaoJogos2013}.

Based on various definitions of games found in the literature and repeated in several introductory articles and also in the observation and detailed analysis of what happens when we play, the following definition of what a game is, can serve as a beacon for our study:

\begin{displayquote}
    Games are social and cultural voluntary activities, meaningful, strongly absorbing, non-productive, that use an abstract world with real-world negotiated effects, and whose development and final outcome is uncertain, where one or more players or teams of players, interactively and quantitatively modify the state of an artificial system, possibly in pursuit of conflicting goals, by means of decisions and actions, some with the capacity to disrupt the opponent, being the whole process regulated, oriented and limited by accepted rules, and thereby obtaining a psychological reward, usually in the form of entertainment, amusement, or a sense of victory over an adversary or challenge.
\end{displayquote}

The reason for having a definition of what a game is is not to create a barrier between games and not games. What we want is to delimit in a fluid way what we consider a game. We may think that this definition is a big \enquote{or} of requisites. Thus, something totally ceases to be a game only when all conditions are broken  \citep{XexeoGeraldoquesaoJogos2013}.

\section{Game Potential}
The high potential that games have in affecting players is mainly due to their ability of placing the player in a continuous mode of interaction, which develops complex cognitive, affective and behavioral responses \citep{YannakakisPlayerModeling2013}.

Every game features at least one user (i.e., the player), who controls an avatar or a group of miniature entities in a virtual\textbackslash{}simulated environment \citep{CallejaIngameimmersionincorporation2011}.
Control may vary from the relatively simple (e.g., limited to movement in an orthogonal grid) to the highly complex (e.g., having to decide several times per second between hundreds of different possibilities in a highly complex 3D world).

The interaction between the player and the game context is of prime importance to modern game development, as it breeds unique stimuli which yield emotional manifestations to the player.

The study of the player in games may not only contribute to the design of improved forms of Human Computer Interactions (HCI)\abbrev{HCI}{Human Computer Interaction}, but also advance our knowledge about human experiences.

\section{Game Industry}
According to the Global Games Market Report \citep{NewGamingBoom} the gaming market has now 2.2 billion gamers across the globe with an expectation of generating \$116 billion in game revenues in 2017 \citep{GlobalGamesMarket}.

It is a competitive market especially in terms of gameplay, which is something largely determined by the quality of a game’s AI.
A game with a large variety of unique agent behaviors has a competitive advantage to a game with a few hand coded expert systems. In addition to being a preferred investment to developers it’s also a viable option on current hardware. Due to hardware advancements, processor cycles are easier to come by and memory is cheap.
